\chapter*{Введение}
\addcontentsline {toc}{chapter}{ВВЕДЕНИЕ}


Человек занимается управлением всю свою сознательную жизнь. Он постоянно для обеспечения желаемого течения тех или иных процессов или сам предпринимает необходимые действия и принимает соответствующие решения, или поручает некоторые из них созданным им системам управления. Свои действия и решения человек выбирает в зависимости от складывающейся ситуации и делает это, как правило, в режиме реального времени, т.е. в таком же темпе, в каком меняется ситуация. С началом современной научно-технической революции стали на базе вычислительной техники создавать и системы управления, функционирующие по принципу управления в реальном времени. Синтез же оптимальных систем управления в реальном времени сдерживается отсутствием быстрых алгоритмов решения задач оптимального управления.

Существуют два взгляда на теорию оптимального управления. Согласно одному из них, возникшему после открытия принципа максимума Понтрягина, теория оптимального управления — раздел современного (неклассического) вариационного исчисления. Другой взгляд трактует теорию оптимального управления как раздел современной теории управления, представляющей естественное развитие классической теории управления. В соответствии с этим в первом случае под управлениями понимаются элементы функциональных пространств, по которым ищется экстремум выбранного функционала качества, и основным вопросом теории считается анализ решения экстремальной задачи (существование, единственность, непрерывная зависимость решений, необходимые и достаточные условия оптимальности и т.п.). 

Во втором случае управление — это процесс, в котором для достижения нужного поведения объекта управления в каждый текущий момент времени создаются целенаправленные (управляющие) воздействия на объект управления в зависимости от доступной к этому моменту информации о поведении объекта и действующих на него возмущениях. Управление называется программным, если (программные) управляющие воздействия (программы) планируются по априорной информации до начала процесса управления и не корректируются в процессе управления. При позиционном управлении (позиционные) управляющие воздействия создаются в процессе управления по текущим позициям, которые аккумулируют информацию, доступную к текущему моменту. Построение оптимальных позиционных управляющих воздействий называется синтезом оптимальных систем управления и является основным вопросом теории оптимального управления во втором случае.

В данной работе будет рассмотрено применение методов параметрического программирования для задачи синтеза оптимальной системы второго порядка.
