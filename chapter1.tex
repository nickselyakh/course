\chapter{Обзор литературы}\label{chap1}


В настоящей главе приводится обзор основных результатов теории оптимального управления. Сначала обсуждаются формулировки задач, возникающих при оптимизации динамических систем управления. Затем приводятся основные результаты качественной теории --- принцип максимума Л.С. Понтрягина и динамическое программирование Р. Беллмана.

В данной главе рассматриваются основные результаты теории оптимального управления, классический подход к посмтроению обратных связей, а также оптимальное управление в реальном времени и реализация оптимальной обратной связи


% %%%%%%%%%%%%%%%%%%%%%%%%%%%%%%%%%%%%%%%%%%%%%%%%%%%%%%%%%%%%%%%%%%%%%%%%%%%%%%%%
% \section{Синтез дискретных систем}\label{1sec:snth}
% %%%%%%%%%%%%%%%%%%%%%%%%%%%%%%%%%%%%%%%%%%%%%%%%%%%%%%%%%%%%%%%%%%%%%%%%%%%%%%%%


% Синтез ДС сводится к выбору управляющего устройства, которое образует с объектом управления некую структуру, позволяющую получить систему с желаемыми характеристиками, под которыми понимаются интересующие разработчика определённые показатели качества. Таким образом, перед разработчиками ставятся задачи синтеза функциональной структуры ДС (приборный синтез) и синтеза корректирующего устройства.
% Приборный синтез состоит в выборе элементов и структуры регулятора, обеспечивающего принцип действия системы. При этом должны обеспечиваться технические условия их нормальной работы при совместном включении. Обычно разработчик на данном этапе использует сведения о параметрах протекающих процессов, физической природе переменных измерения и управления, а также имеющихся технических средствах. Полученные на этом этапе объект или часть ДС называют неизменяемой системой. Этот этап проектирования не имеет пока строгой математической основы и относится к области инженерного искусства.
% Корректирующий синтез подразумевает введение таких добавочных устройств, которые улучшат качество работы неизменяемой системы и при этом не нарушат принципиальной работоспособности устройства. Из теории непрерывных систем управления известно, что коррекция может быть последовательная, параллельная, коррекция за счёт введения обратной связи и смешанная. С математических позиций все они эквивалентны, так как корректирующему контуру одного типа может быть сопоставлен корректирующий контур другого типа.
% Важнейшим этапом проектирования ДС является разработка цифровых законов управления непрерывными объектами. Для решения этой задачи предложено три подхода: 1) переоборудование, которое сводится к замене непрерывного регулятора его дискретной моделью в результате аппроксимации; 2) дискретизация объекта путём построения дискретной модели непрерывного объекта и последующий синтез регулятора методами теории дискретных систем; 3) прямой синтез цифрового регулятора для непрерывного объекта без каких-либо упрощений и аппроксимаций.
% Первые два подхода являются приближёнными и фактически предполагают замену одной задачи другой с целью применить известные результаты теории стационарных (непрерывных или дискретных) систем.   В первом подходе игнорируется наличие цифровой части (импульсного элемента, дискретного регулятора и экстраполятора) с помощью дискретного преобразования Лапласа и используются методы синтеза непрерывных систем. При этом иногда дискретизация полученного аналогового регулятора не позволяет добиться желаемого эффекта. При использовании

% второго подхода не учитывается поведение системы в промежутках между моментами квантования, что может привести к появлению скрытых колебаний и несоответствию качества полученной системы заданным критериям. Другой недостаток метода дискретизации объекта состоит в том, что требования к системе, сформулированные в непрерывном времени, не всегда легко перевести в соответствующие дискретизированные показатели качества. На современном этапе в теории дискретных систем управления основное внимание уделяется точным методам анализа и синтеза. Во многом это связано с тем, что приближённые методы проектирования могут приводить к неработоспособным решениям.
% Применительно к первым двум подходам разработаны определённые методы синтеза, которые позволяют получить выражение для реализации регулятора. К этим методам относятся [13]:
% 	•	графоаналитические методы инженерного динамического синтеза САУ: корневые, корневого годографа, стандартных переходных характеристик, частотные;
% 	•	аналитические методы оптимального и адаптивного управления;
% 	•	синтез САУ по интегральным критериям качества (функционалам) методами вариационного исчисления, динамического линейного и нелинейного математического программирования, принципом максимума Понтрягина, методом Винера-Хопфа, методами модального управления, методами аналитического конструирования оптимальных регуляторов и др.
% Для третьего подхода применяются ЭВМ с использованием специализированного программного обеспечения на основе методов аналогового и цифрового моделирования САУ. По сравнению с первыми двумя методами здесь реализуется наиболее полное исследование свойств и синтез САУ с учётом  всех  особенностей  объекта  управления.  Аналитические и графоаналитические методы, в свою очередь, позволяют исследовать САУ в более общем виде и найти оптимальный вариант среди множества решений.

%%%%%%%%%%%%%%%%%%%%%%%%%%%%%%%%%%%%%%%%%%%%%%%%%%%%%%%%%%%%%%%%%%%%%%%%%%%%%%%%
\section{Основные результаты теории оптимального управления}\label{1sec:optimal-control}
%%%%%%%%%%%%%%%%%%%%%%%%%%%%%%%%%%%%%%%%%%%%%%%%%%%%%%%%%%%%%%%%%%%%%%%%%%%%%%%%

Постановка любой конкретной задачи оптимального управления включает в себя 5 необходимых элементов: промежуток управления, математическую модель управляемого объекта, класс управлений и ограничений на них, ограничения на фазовую траекторию,
критерий качества.
Рассмотрим их подробнее.

1) Промежуток управления. Прежде всего задачи оптимального управления разделяются на непрерывные, рассматриваемые на некотором промежутке времени $T = [t_{0},t_{f}]$, и дискретные, в которых динамический процесс рассматривается в дискретные моменты времени $k = 0,1,...N$, где $N$ — натуральное число.

По продолжительности процесса различаются задачи с фиксированным и нефиксированным временем окончания процесса. Выделяются также задачи на бесконечном интервале.

2) Математическая модель. Динамика изучаемого процесса моделируется, как правило, дифференциальными (для непрерывных систем)

\begin{equation}\label{1v}\dot{x}(t) = f(x(t),u(t),t), t \in [t_0,t_f],\end{equation}
или разностными уравнениями (для дискретных систем)

$$x(k + 1) = f(x(k),u(k),k),k = 0,1,...,$$
где $n$-вектор $x$ называется состоянием системы, $r$-вектор $u$ называется
управлением, функция 
$ f : R \times\ R^r \times R \rightarrow R^n$ задана.

Число переменных состояния $n$ называется порядком системы управления, число $r$ — числом входов.

Далее будем рассматривать непрерывные системы вида (\ref{1v}).

3) Класс управлений и ограничения на них. Для непрерывного процесса управления указывается класс функций, из которого выбираются управления. Это могут быть: измеримые, дискретные, кусочно-непрерывные, гладкие, импульсные функции и т.д.

Кроме класса доступных управлений задается множество 
$U \subset\ R$ — множество допустимых значений управления. Как правило, $U$ — компакт в $R^r$.

Далее будем рассматривать управления из класса кусочно-непрерывных функций.

\begin{definition}Кусочно-непрерывная функция $u(\cdot) = (u(t),t \in [t_{0},t_{f}])$ называется доступным управлением, если $u(t) \in U, t \in [t_{0},t_{f}].$
\end{definition}

4) Ограничения на фазовую траекторию. Ограничения на переменные состояния могут накладываться:\begin{itemize}
\item в начальный момент времени $t_{0}$:
$$x(t_{0}) \in X_{0};$$
\item в конечный момент времени $t_{f}$ --- такие ограничения называются терминальными: $$x(t_{f}) \in X_{f};$$
\item в изолированные моменты $t_{i} \in [t_{0},t_{f}], i = \overline{1,m},$ из промежутка управления — промежуточные фазовые ограничения:
$$X(t_{i}) \in X_{i},i = \overline{1,m},$$
\item на всем промежутке управления — фазовые ограничения:
$$x(t) \in X(t),t \in [t_{0},t_{f}],$$ где $X_{0}, X_{f}, X_{i}, i = \overline{1,m}, X(t), t \in [t_0,t_f],$ — заданные подмножества пространства состояний.
\end{itemize}
 Задача управления с $x(t_f) \in X_f $ называется:\begin{itemize}
  \item задачей со свободным правым концом траектории, если
 $X_f = R^n$, \item задачей с закрепленным правым концом траектории, если $X_{f} = \{x_{f}\}$, \item задачей с подвижным правым концом траектории, если $X_{f}$ содержит более одной точки и не совпадает с $R^{n}$.
\end{itemize}
 Аналогичная классификация имеет место для задач с ограничениями на левый конец траектории $x_{0} \in X_{0}$.

  Выделяют также смешанные ограничения, учитывающие связи между переменными состояния и переменными управления:
$$(u(t),x(t)) \in S \subseteq R^r \times R^n,t \in [t_{0},t_{f}[.$$ \begin{definition} Доступное управление $u(\cdot)$ называется допустимым (или, программой), если оно порождает траекторию $x(\cdot)=(x(t), t \in [t_0,t_f]$, удовлетворяющую всем заданным ограничениям задачи. \end{definition}

5) Критерий качества. Качество допустимого управления оценивается так называемым критерием качества

 Существуют четыре типа критерия качества:

i) критерий качества Майера (терминальный критерий)
$$J(u) = \varphi(x(t_{f})),$$

ii) критерий качества Лагранжа (интегральный критерий)
$$J(u) =\int^{t_{f}}_{ t_{0}}
f_{0}(x(t),u(t),t)dt,$$

iii) критерий качества Больца
$$J(u) = \varphi(x(t_{f})) +
\int^{t_{f}}_{ t_{0}}
f_0(x(t),u(t),t)dt,$$

iv) задачи быстродействия (являются задачами с нефиксированной продолжительностью процесса).
$$J(u) = t_{f} - t_{0} \rightarrow \min.$$
\begin{definition} Допустимое управление $u^{0}(\cdot)$ называется оптимальным управлением (оптимальной программой), если на нем критерий качества достигает экстремального значения (min или max):
$$J(u^0) = extr J(u),$$

где минимум (максимум) берется по всем допустимым управлениям.
\end{definition}
%%%%%%%%%%%%%%%%%%%%%%%%%%%%%%%%%%%%%%%%%%%%%%%%%%%%%%%%%%%%%%%%%%%%%%%%%%%%%%%%
\section{Принцип максимума и динамическое программирование}\label{1sec:pm} % xxx заменить на свою метку
%%%%%%%%%%%%%%%%%%%%%%%%%%%%%%%%%%%%%%%%%%%%%%%%%%%%%%%%%%%%%%%%%%%%%%%%%%%%%%%%
В теории оптимального управления существует два фундаментальных результата: принцип максимума Л.С. Понтрягина\cite{Pontryagin} и динамическое программирование Р. Беллмана.\cite{Bellman} Приведем эти результаты на примере простейшей задачи оптимального управления.

 $$J(u) = \varphi(x(t_{f})) +
\int^{t_{f}}_{ t_{0}}
f_0(x(t),u(t),t)dt \rightarrow \min, $$
 \begin{equation}\label{tr2}\dot{x}(t)=f(x(t),u(t),t), x(t_0)=x_0,\end{equation}
$$u(t) \in U, t\in [t_0,t_f].$$

\subsection{Принцип максимума Понтрягина}

Принципом максимума называется основное необходимое условие оптимальности в задачах оптимального управления, связанное с максимизацией гамильтониана:

$$H(x,\psi, u, t) = \psi 'f(x, u, t) - f_0(x, u, t) = \sum^n_{j=1} \psi_j f_j(x, u, t) - f_0(x, u, t).$$ Здесь $\psi = \psi (t) \in R $ --- сопряженная переменная.
\begin{theorem}Пусть $u^0(\cdot), x^0(\cdot)$ — оптимальное управление и траектория в задаче (\ref{tr2}), $\psi^0(\cdot)$ — соответствующее решение сопряженной системы
$$ \dot{\psi}^0(t)= -\frac{\partial H}{\partial x}(x^0(t),\psi^0(t),u^0(t),t),$$ с начальным условием
$$\psi^0(t_f) = - \frac{\partial\varphi}{\partial x}(x^0(t_f)). $$
Тогда для любого $t \in [t_0,t_f]$, управление $u^0(t)$ удовлетворяет условию:
$$H(x^0(t),\psi^0(t),u^0(t),t) = \max_{v\in U}
H(x^0(t),\psi^0(t),v,t), t \in [t_0,t_f].$$
\end{theorem}

Для того чтобы решить задачу с помощью принципа максимума обычно поступают следующим образом. Функцию $H(x, \psi ,u,t)$ рассматривают как функцию $r$  переменных $u = (u_1,...,u_r).$ Далее проводят поточечную оптимизацию для каждого фиксированного набора $(x, \psi ,t)$
\begin{equation}\label{krz}u(x, \psi ,t) = \arg \max_{v \in U} H(x, \psi , v, t).\end{equation}
Если исходная задача (\ref{tr2}) имеет решение, функция (\ref{krz}) определена на непустом множестве значений $(x, \psi , t).$

Пусть  $u$ в виде (\ref{krz}) найдена, тогда можно рассмотреть следующую систему с граничными условиями:
$$ \dot{x} =\frac{\partial{H}}{\partial{\psi}}(x, \psi , u(x, \psi , t), t) = f(x, \psi , u(x, \psi , t), t), \ x(t_0) = x_0,$$
$$ \dot{\psi } = - \frac{\partial{H}}{\partial{x}}(x, \psi , u(x, \psi , t), t), \  \psi (t_f) = -\frac{\partial{\varphi(x(t_f))}}{\partial{x}}.$$
Таким образом получена специальная краевая задача, которая называется краевой задачей принципа максимума.

Можно ожидать, что имеются лишь отдельные изолированные пары функций $x(\cdot), \psi (\cdot),$ удовлетворяющие краевой задаче принципа максимума. Подставив одну такую пару в (\ref{krz}), получим:
\begin{equation}\label{prtn}u(t) = u(x(t), \psi (t),t), t \in [t_0, t_f],\end{equation}
которая удовлетворяет принципу максимума и, значит, может претендовать на роль оптимального управления, а функция $x(t) = x(t\mid t_0,x_0,u(\cdot)), t\in [t_0, t_f],$ --- на роль оптимальной траектории в задаче.

Отметим, что принцип максимума в задаче (\ref{tr2}) является лишь необходимым условием оптимальности, поэтому построенное управление не может быть оптимальным. Построенная функция называется экстремалью Понтрягина (\ref{prtn}).
\subsection{Динамическое программирование}

Рассмотрим задачу (\ref{tr2}) и предположим, что она имеет решение. Следуя динамическому программированию\cite{Bellman}, погрузим задачу (\ref{tr2}) в семейство задач
$$ J_{\tau ,z}(u) = \varphi (x(t_f)) + \int^{t_{f}}_{\tau }
f_0(x(t),u(t),t)dt \rightarrow \min, $$
 \begin{equation}\label{dp1}\dot{x}(t)=f(x,u), \  x(\tau )=z,\end{equation}
$$u(t) \in U, \ t\in T = [\tau ,t_f],$$
зависящих от скаляра $\tau \in T$ и $n$-вектора $z.$

Пару $(\tau , z)$ назовем позицией в задаче (\ref{tr2}). Обозначим через $$B(\tau ,z) = \min J_{\tau ,z} (u)$$
минимальное значение критерия качества в задаче (\ref{dp1}) для позиции $(\tau , z)$. Если для позиции  $(\tau , z)$ задача (\ref{dp1}) не имеет решения, положим $B(\tau ,z) =  +\infty$. Пусть $$X_{\tau } = \{z \in R : B(\tau ,z) < +\infty \}.$$
Функцию\begin{equation}\label{fb} B(\tau ,z), z \in X_{\tau}, \tau \in T,\end{equation} называют функцией Беллмана.

Уравнение в частных производных, которому удовлетворяет функция (\ref{fb}), называют уравнением Беллмана
\begin{equation}\label{ub} -\frac{\partial B(\tau,z)}{\partial\tau} = \min_{v \in U}\left\{ \frac{\partial B'(\tau,z)}{\partial z}f(z,v) + f_0(z,v)\right\}, z \in X_\tau, \tau \in T.\end{equation}
Выделяя из семейства (\ref{dp1}) задачу с $\tau = t_f$, находим граничное условие для уравнения Беллмана
\begin{equation}\label{gusl}B(t_f,z)=\begin{cases}
\varphi(z), если z \in X, \\
+\infty , z \not\in X.
\end{cases}\end{equation}
\bigskip


%%%%%%%%%%%%%%%%%%%%%%%%%%%%%%%%%%%%%%%%%%%%%%%%%%%%%%%%%%%%%%%%%%%%%%%%%%%%%%%%
\section{Классический подход к построению оптимальных обратных связей}\label{1sec:MPC}
%%%%%%%%%%%%%%%%%%%%%%%%%%%%%%%%%%%%%%%%%%%%%%%%%%%%%%%%%%%%%%%%%%%%%%%%%%%%%%%%

\begin{equation} \label{1problem}
    J(u) = \varphi(x(t_f))\to \min,
    \end{equation}
\begin{equation} \label{2problem}
    \dot{x}=f(x,u,t),\ x(t_0)=x_0
     \end{equation}
\begin{equation} \label{3problem}
  	x(t)\in X\in\mathbb{R}^n
     \end{equation}
\begin{equation} \label{4problem}
  	 u(t)\in U\in\mathbb{R}^r,\  t\in [t_0, t_f]
     \end{equation} 


В поставленной задаче нужно минимаизировать критерий качества (\ref{1problem}) на траекториях системы (\ref{2problem}), которые в каждый момент времени лежат в заданном множестве X (\ref{3problem}) с помощью ограниченных управляющих воздействий (\ref{4problem}).
(\ref{1problem}) - (\ref{4problem}):  $t_0, t_f$ - заданы,\\
$x = x(t)\in\mathbb{R}^n$ --- состояния системы управления в t,\\
$u = u(t)\in\mathbb{R}^r$ --- значения управляющего воздействия в t.\\
$f:\mathbb{R}^n\times\mathbb{R}^r\times\mathbb{R}^n$ обеспечивает существование и продолжимость решения уравнения (\ref{2problem}) на промежутке времени $T = [t_0; t_f]$. \\

Задачу (\ref{1problem}) - (\ref{4problem}) будем рассматривать в классе кусочно-непрерывных управляющих воздействий u.
Допустимое программное управление (программа) --- кусочно-непрерывная функция $u(\cdot)\in U$,
если она порождает траекторию х системы (\ref{2problem}), удовлетворяющей (\ref{3problem}). Допустимое программное управление - оптимальное (оптимальная программа), если на нем критерий качества (\ref{1problem}) достигает оптимального значения.
$J(u^0) = min J(u)$, где минимум берется по всем программам.\\
Для введения понятия классической оптимальной обратной связи погрузим задачу (\ref{1problem}) - (\ref{4problem}) в следующее семейство задач:

\begin{equation} \label{5problem}
    \varphi(x(t_f))\to \min,
    \dot{x}=f(x,u,t),\ x(z)=z
  	x(t)\in X\\
  	 u(t)\in U,\  t\in [\tau, t_f]
     \end{equation}

Пусть $u^0(t|\tau,z),\ t\in T_\tau$ --- оптимальная программа задачи (\ref{5problem}) для позиции $(\tau,z)$ \\
$X_\tau$ - множество состояний z таких, что для позиции $(\tau,z)$ существуют программные решения задачи (\ref{5problem}) \\
Функция
\begin{equation} \label{6problem}
    u^0(\tau,z) = u^0(t|\tau,z)
     \end{equation}
     --- оптимальная обратная связь.\\

Построение (\ref{6problem}) --- синтез оптимальной системы управления.\\
Подстановка функции (\ref{6problem}) в уравнение (\ref{3problem}) --- замыкание системы управления
\begin{equation} \label{7problem}
    \dot{x}=f(x,u^0(t,x),t),\ x(t_0)=x_0
     \end{equation}
Полученное уравнение - математическая модель оптимальной автоматической системы управления.\\
Для любого состояния $x_0$ решение уравнения (\ref{7problem}) с начальным состоянием $x(t_0)=x_0$ - оптимальная траектория для задачи (\ref{1problem}) - (\ref{4problem}).\\
Отметим, что если оптимальная обратная связь построена в классе кусочно-непрерывных управлений, то во многих задачах уравнение (\ref{7problem}) представляет собой
дифференциальное уравнение с разрывной правой частью и, вообще говоря, не имеет классического решения. В этих случаях используют обобщенное решение.\\
Чтобы избежать аналитических трудностей, таких как определение решения замкнутой системы управления, в задаче оптимального управления (\ref{1problem}) - (\ref{4problem})
перейдем от класса кусочно-непрерывных доступных управлений к дискретным управлениям.\\
Разобьем Т на $n\in N$ частей $h = \frac{t_f - t_0}{N}$

\begin{equation}
    u(t) = u(s),\ t\in [s, s+h],\ s\in T_h = \{t_0, t_0 + h, ..., t_f - h\}
     \end{equation}
Понятие программы и оптимальной программы в классе дискретных управлений аналогично определению в классе кусочно-непрерывных управлений.\\
Семейство, в которое прогружается задача (\ref{1problem}) - (\ref{4problem}) будет выглядеть следующим образом:
\begin{equation} \label{8problem}
    \varphi(x(t_f))\to \min,
    \dot{x}=f(x,u,t),\ x(z)=z,\ 
  	 u(t)\in U,\  t\in T_h.
     \end{equation}

Семейство зависит от $(\tau,z), z\in \mathbb{X},\ \tau\in T_h$\\
Оптимальная дискретная обратная связь --- функция 
\begin{equation} \label{9problem}
    u^0(\tau,z) = u^0(t|\tau,z),\  z \in X_\tau,\ \tau \in T_h
\end{equation}

\section{Оптимальное управление в реальном времени и реализация оптимальной обратной связи}\label{1sec:real-time}
Проанализируем, как используется $u^0(\tau,z) = u^0(t|\tau,z),\ \tau\in T_h,\ z\in \mathbb{X_\tau}$\\
Продолжая процесс таким образом получим управляющее воздействие $u^*(t), t\in T$ и траекторию $x^*(t), t\in T$ удовлетворяющее тождеству
$x^*(t) \equiv f(x^*(t), u^*(t), t) + w^*(t), t\in T, x^*(t_0) = x_0$
\begin{equation} \label{9problem}
    u^*(t), t\in T: u^*(t) = u^0(\tau,x^*(z)), t\in [\tau, \tau + h], \tau\in T_h.
     \end{equation}
--- реализация оптимальной обратной связи (\ref{1problem}) в конкретном процессе управления.\\
Отсюда видно, что в конкретном процессе управления оптимальная обратная связь (\ref{1problem}) не используется целиком на всей области ее определения.
Нужны лишь ее значения вдоль одной последовательности измеряемых состояний физического объекта $x^*(\tau), \tau\in T_h$, и достаточно уметь
для каждой текущей позиции $(\tau,x^*(\tau)), \tau\in T_h$, вычислять значение реализации 
$u^*(t) = u^0(\tau, x^*(\tau)),\ t\in [\tau, \tau + h]$ оптимальной обратной связи за время $s(\tau)<h$\\
Будем говорить, что процесс управления осуществляется в реальном времени, если для каждого текущего момента $\tau\in T_h$ зачение $u^*(\tau)$
вычисляется за время $s(\tau)<h$, т.е. до получения следующего измерения $x^*(\tau+h)$. Устройство, способное реализовать оптимальную обратную связь в реальном времени
будем называть оптимальным регулятором.\\
Таким образом, с использованием принципа управления в реальном времени задача синтеза ОУ типа ОС сводится к построению алгоритма работы оптимального регулятора.


%%%%%%%%%%%%%%%%%%%%%%%%%%%%%%%%%%%%%%%%%%%%%%%%%%%%%%%%%%%%%%%%%%%%%%%%%%%%%%%%
\section{Общие результаты для многопараметрических нелинейных программ}\label{2sec:results-multy-parametric}
%%%%%%%%%%%%%%%%%%%%%%%%%%%%%%%%%%%%%%%%%%%%%%%%%%%%%%%%%%%%%%%%%%%%%%%%%%%%%%%%

ДОПИСАТЬ